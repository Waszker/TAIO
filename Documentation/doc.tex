\documentclass{llncs}
\usepackage[polish]{babel}
\usepackage[utf8]{inputenc}
\usepackage{polski}
\frenchspacing
\setcounter{tocdepth}{2}

\begin{document}
	
\begin{titlepage}

\newcommand{\HRule}{\rule{\linewidth}{0.5mm}}

\center

%----------------------------------------------------------------------------------------

\textsc{\LARGE Politechnika Warszawska}\\[5mm]
\textsc{\LARGE Wydział Matematyki i Nauk Informacyjnych}\\[4cm]
 
%----------------------------------------------------------------------------------------

\textsc{\Large Teoria Algorytmów i Obliczeń}\\[0.5cm]

%----------------------------------------------------------------------------------------

\HRule \\[0.4cm]
{ \huge \bfseries Konstrukcja automatu deterministycznego skończonego sprawdzającego zachodzenie relacji indukowanej przez język dla słów z danego języka (teoretyczne opracowanie zadania)}\\[0.4cm]
\HRule \\[1.5cm]
 
%----------------------------------------------------------------------------------------

\begin{flushright}
\Large \emph{Autorzy:}\\[0.5cm]
Sylwia \textsc{Nowak}\\
Anna \textsc{Zawadzka}\\
Pavel \textsc{Kuzmich}\\
Piotr \textsc{Waszkiewicz}\\
\end{flushright}

%----------------------------------------------------------------------------------------

\vfill
{\large \today}\\[3cm]

\end{titlepage}
	
\tableofcontents
\newpage

\section{Wstęp}

\subsection{Cel projektu}

Celem projektu jest stworzenie deterministycznego automatu skończonego realizującego funkcję relacji indukowanej przez język na podstawie odpowiedzi udzielanych przez gotowe, nieznane narzędzie pełniące tę samą rolę.

\subsection{Sposób realizacji w etapach}

\begin{itemize}
\item[•] Wprowadzenie narzędzia (automatu), pozwalającego sprawdzić czy $\forall_{x,y \in \Sigma^*} x R_{L} y$.
\item[•] Utworzenie treningowego i testowego zbioru słów.
\item[•] Znalezienie skończonych automatów deterministycznych spełniających cel projektu z wykorzystaniem algorytmu PSO (Particle Swarm Optimization) i zbioru treningowego słów.
\item[•] Wybranie najlepszego utworzonego automatu wykorzystując zbiór testowy słów.
\end{itemize}

\newpage

\section{Szczegółówy opis sposobu realizacji zadania}

\subsection{Automat wejściowy}

Na podstawie automatu wejściowego zostanie stworzone narzędzie sprawdzające, czy dwa słowa są ze sobą w relacji. Informacje o automacie wejściowym będą przechowywane w pliku tekstowym podawanym na wejście programu. Format reprezentacji automatu jest następujący:

\begin{itemize}
\item[•] Stany numerowane liczbami 1, 2, …, n, gdzie n to liczba stanów, 1 – stan początkowy.
\item[•] $r$ – liczba symboli w alfabecie.
\item[•] Tabela funkcji przejścia.
\end{itemize}

Po wprowadzeniu pliku z opisem automatu będzie można zobaczyć jego graficzną reprezentację w postaci grafu.

\subsection{Zbiór treningowy i zbiór testowy}

Utworzenie treningowego i testowego zbioru słów nastąpi poprzez wygenerowanie odpowiednich zbiorów. Będą one dostępne (nie zmienią się) dopóki zbiór symboli alfabetu pozostanie ten sam. Zbiór treningowy zawierać będzie wszystkie słowa nad alfabetem o długości mniejszej bądź równej 5 symboli oraz wariacje bez powtórzeń dla słów o długości 6.\\

Zbiór treningowy i testowy są zbiorami rozłącznymi.\\

Zbiór testowy składać się będzie ze słów będących wariacjami bez powtórzeń dla słów o  długości nie mniejszej niż 7 i nie większej niż maksymalna ilość stanów poszukiwanego automatu. W przypadku mniejszej ilości symboli alfabetu niż wymagałyby tego wariacje następuje konstrukcja nowego, tymczasowego zbioru poprzez powielenie wszystkich symboli alfabetu i dodanie ich do istniejących już symboli. Nowe symbole będą dodatkowo indeksowane numerem oznaczających w której iteracji zostały dodane. Zapewni to ich rozróżnialność podczas konstrukcji zbiorów.\\

Przykład zbioru: {0, 1}. 
Zbiór po przygotowaniu do utworzenia wariacji bez powtórzeń dla słów długości 6: {0, 1, $0_{1}$, $1_{1}$, $0_{2}$, $1_{2}$}.


\subsection{Opis przestrzeni poszukiwania dla algorytmu PSO}

Przestrzeń poszukiwania w algorytmie PSO będą stanowić wszystkie deterministyczne automaty skończone o określonej liczbie stanów, z alfabetem wejściowym równoważnym z alfabetem języka $L$. Zapis cyfrowy każdego automatu można przedstawić jako zbiór macierzy binarnych o wymiarach $|Q|^2$ (gdzie $|Q|$ to ilość stanów automatu). Zbiór ten jest równoliczny ze zbiorem symboli alfabetu automatu (każdemu symbolowi odpowiadać będzie inna macierz). Jeżeli jako dodatkowe założenie przyjmiemy, że w każdym wierszu każdej macierzy znajdować może się tylko jedna jedynka, to wtedy taką macierz traktować możemy jako część tabelki funkcji przejścia automatu, która dla wiersza o numerze i oraz jedynce w kolumnie $j$ odpowiada przejściu ze stanu $q_{i}$ do stanu $q_{j}$ po wczytaniu symbolu przypisanego do tej macierzy. Ponieważ następująca reprezentacja jest nieoptymalna pod względem zużycia pamięci zapis można uprościć do tablicy jednowymiarowej która dla każdego indeksu odpowiadającego numerowi stanu w którym automat znajduje się przed wczytaniem symbolu, przyporządkowuje numer stanu w którym automat znajdzie się po wczytaniu elementu z taśmy.\\

Dla przykładu, po zastosowaniu takiej optymalizacji dla tabeli funkcji przejścia 
\begin{table}[]
\centering
\begin{tabular}{c|c|c|c|}
symbol                   & $q_{1}$ & $q_{2}$ & $q_{3}$ \\ \hline
\multicolumn{1}{c|}{$q_{1}$} & 0  & 1  & 0  \\ \hline
\multicolumn{1}{c|}{$q_{2}$} & 1  & 0  & 0  \\ \hline
\multicolumn{1}{c|}{$q_{3}$} & 0  & 0  & 1  \\ \hline
\end{tabular}
\vspace{0.5cm}
\caption{Przykładowa tabela funkcji prześcia}
\end{table}

otrzymamy jako wynik wektor [1 0 2].\\

Przyjmijmy $d = |\Sigma|$ liczba symboli alfabetu, $f = |Q|$ liczba stanów automatu. Tak więc przestrzeń wszystkich automatów stanowić będą wszystkie możliwe ciągi liczb o długości $d*f$ i wartościach z przedziału $[0, f)$. Cząstki poruszające się w przestrzeni będą przyjmowały pozycje będące reprezentacją tych automatów. Dla każdej cząsteczki jej prędkość definiowana będzie jako $V_{p} = (V_{1}, V_{2}, …, V_{d})$ gdzie $V_{i (1 \leq i \leq d)}$ oznacza prędkość cząstki na płaszczyźnie odpowiadającej symbolowi o numerze $i$, definowanej jako $V_{pi} = (V_{1}, V_{2}, …, V_{f})$. $V_{j (1 \leq j \leq f)}$ reprezentuje tutaj prędkość cząstki dla wiersza numer $j$ w macierzy zdefiniowanej wcześnie jako reprezentacja tabeli funkcji przejścia dla danego symbolu. Tak zdefiniowaną prędkość można (podobnie jak położenie) przedstawić w postaci tablicy jednowymiarowej o długości $d*f$ (gdzie $d$ i $f$ - zdefiniowane wcześniej). Zmiana położenia cząstki następuje więc poprzez dodanie wektora prędkości do wektora położenia tym samym otrzymując nowe położenie.

\subsection{Odległość w przeszukiwanej przestrzeni}

Dla tak zdefiniowanej przestrzeni i ciągów liczb długości d*f, które reprezentować będą automaty wprowadzimy metrykę, nazywaną metryką Hamminga. Dla dwóch ciągów $a$ i $b$ z tej przestrzemi odległość Hamminga jest równa liczbie jedynek w słowie $a XOR b$, np. odległość pomiędzy ciągami 10\textbf{0}11\textbf{1}01 i 10\textbf{1}11\textbf{0}01 wynosi 2.

\subsection{Przeszukiwanie przestrzeni w algorytmie PSO}

Przeszukiwanie przestrzeni automatów realizowane jest poprzez wygenerowanie pewnej ilości cząstek (nazywanych rojem) poruszających się według określonych zasad i dążących do osiągnięcia postawionego celu. Za cel najczęściej przyjmuje się minimalizację wartości pewnej funkcji nazywanej funkcją celu. W tym projekcie będzie to funkcja $f:A \rightarrow N$ ($A$ - zbiór automatów, $N$ - liczby naturalne), która dla danego automatu zwraca liczbę błędnie zaklasyfikowanych słów.\\

W trakcie kolejnych kroków zdyskretyzowanego czasu cząstki przemieszczają się do nowych pozycji, symulując adaptację roju do środowiska poszukując optimum. “Liderem” roju zostaje cząstka o najlepszym dotychczas znanym położeniu. Dla każdej cząstki możemy przyporządkować pewnych “sąsiadów”, którymi są wybrane cząstki z roju. Sąsiedzi określani będą raz, na początku obliczeń. “Liderem sąsiadów” dla każdej cząstki jest sąsiad, o najlepszym dotychczas znalezionym położeniu. Położenia cząstek w obszarze przeszukiwania stanowią potencjalne rozwiązania.\\

Algorytm PSO funkcjonuje według następującego schematu:

\begin{enumerate}
\item Nadanie cząstkom roju losowych położeń.
\item Dokonanie oceny położenia cząsteczek za pomocą funkcji dopasowania (fitness).
\item Zmiana zapisu w pamięci cząstek dotycząca najlepszych własnych położeń oraz najlepszych położeń liderów sąsiadów. Wyłonienie lidera roju.
\item Aktualizacja wektora prędkości każdej cząstki

\begin{displaymath}
v_{k+1}^i = w_{1k}^i v_{k}^i + w_{2k}^i[c_{1} r_{1k}^i(p_{k}^{li} - x_{k}^i) + c_{2} r_{2k}^i(p_{k}^{g} - x_{k}^i) + c_{3} r_{3k}^i(p_{k}^{ni} - x_{k}^i)] \hspace{15mm} (1)
\end{displaymath}

\item Aktualizacja położenia każdej cząstki

\begin{displaymath}
x_{k+1}^i = x_{k}^i + v_{k+1}^i \hspace{15mm} (2)
\end{displaymath}

gdzie

\begin{itemize}
\item[$x_{k}^i$] -- wektor położenia $i$-tej cząstki,
\item[$v_{k}^i$] -- odpowiedni wektor prędkości,
\item[$p_{k}^{li}$] -- najlepsze dotąd znalezione położenie $i$-tej cząstki,
\item[$p_{k}^{ni}$] -- najlepsze dotąd znalezione położenie lidera sąsiadów $i$-tej cząstki,
\item[$p_{k}^g$] - najlepsze dotąd znalezione położenie lidera roju,
\item[$w_{1k}^i, w_{2k}^i$] - współczynniki wagowe, określane na poziomie $i$-tej cząstki.
\end{itemize}
\end{enumerate}

W powyższych oznaczeniach dolny indeks k określa kolejny krok iteracji. Współczynniki $c_{1}, c_{2}, c_{3}$ są ustalonymi mnożnikami wagowymi. Współczynniki  są liczbami losowymi o rozkładzie równomiernym w przedziale [0, 1]. Generowane są w kolejnych krokach iteracji dla każdej cząstki.\\

Wielkość $w_{1k}^i$ jest mnożnikiem interpretowanym jako współczynnik bezwładności ruchu cząstki. Wartości współczynników wagowych dobierane są w zależności od zachowania się cząstek roju. Jeżeli cząstka przemieściła się do lepszego położenia, czyli nastąpiła poprawa wartości funkcji celu, to w kolejnym kroku wartość $w_{2k}^i$  przyjmuje się jako równą 0, natomiast wartość $w_{1k}^i$ $\geq$ 1, zezwalając na swobodny ruch w tym kierunku i dalsze przyspieszanie cząstki (stan 1). Gdy cząstka nie poprawiła swego dopasowania, wartości obu współczynników przyjmuje się równe 1 i nowe położenie jest określane z wykorzystaniem historii ruchu (stan 2).\\

Punkty 2-5 należy wykonywać w pętli dopóki oba warunki będą spełnione:

\begin{itemize}
\item[•] któraś z cząstek nie osiągnie założonej wartości minimalnej błędu,
\item[•] nie zostanie osiągnięta z góry wyznaczona ilość iteracji ruchu.
\end{itemize}

Jeden przebieg pętli odpowiada krokowi czasowemu dla poruszającego się roju. Wartość kroku czasowego przyjmuje się jako jednostkową.
Cząstki poruszają się ruchem odcinkowo prostolinijnym. Postać równań (1) i (2) wynika z drugiej zasady dynamiki Newtona, gdzie wypadkowa siła wywołująca przyspieszenie każdej cząstki powstaje wskutek działania sił od “naciągniętych sprężyn” pomiędzy położeniami aktualnym a najlepszymi położeniami: własnym, lidera sąsiada oraz lidera roju. Kolejny wektor prędkości czasu jest wynikiem jej przyspieszenia w kierunku nowego, potencjalnie lepszego położenia, bazując na stale zmieniających się najlepszych wcześniej znanych położeniach.\\

Dla każdej wartości $i$ = 1, 2, 3, …  algorytm PSO uruchamiany będzie w przestrzeni z automatami o $i$ stanach. Warto zwrócić uwagę, że istnieje możliwość znalezienia kilku automatów spełniających warunki zakończenia algorytmu PSO. Aby zminimalizować ryzyko wybrania niewłaściwego automatu, po zakończeniu obliczeń dla każdej wartości parametru $i$ najlepszy automat dla danej ilości stanów będzie testowany na zbiorze testowym. Ostateczna decyzja wyboru najlepszego rozwiązania podjęta będzie na podstawie uzyskanej wartości funkcji celu na tym zbiorze.

\end{document}