
%%%%%%%%%%%%%%%%%%%%%%% file typeinst.tex %%%%%%%%%%%%%%%%%%%%%%%%%
%
% This is the LaTeX source for the instructions to authors using
% the LaTeX document class 'llncs.cls' for contributions to
% the Lecture Notes in Computer Sciences series.
% http://www.springer.com/lncs       Springer Heidelberg 2006/05/04
%
% It may be used as a template for your own input - copy it
% to a new file with a new name and use it as the basis
% for your article.
%
% NB: the document class 'llncs' has its own and detailed documentation, see
% ftp://ftp.springer.de/data/pubftp/pub/tex/latex/llncs/latex2e/llncsdoc.pdf
%
%%%%%%%%%%%%%%%%%%%%%%%%%%%%%%%%%%%%%%%%%%%%%%%%%%%%%%%%%%%%%%%%%%%


\documentclass[runningheads,a4paper]{llncs}
\usepackage[margin=0.5in]{geometry}
\usepackage{amssymb}
\setcounter{tocdepth}{3}
\usepackage{graphicx}
\usepackage[polish]{babel}
\usepackage[utf8]{inputenc}
\usepackage{polski}
\usepackage{color}

\newcommand{\keywords}[1]{\par\addvspace\baselineskip
\noindent\keywordname\enspace\ignorespaces#1}

\begin{document}
\vspace{-100pt}
\mainmatter  % start of an individual contribution
% first the title is needed
\title{Konstrukcja automatu deterministycznego skończonego sprawdzającego zachodzenie relacji indukowanej przez język dla słów z danego języka (dokumentacja uzupełniająca)\\Teoria algorytmów i obliczeń}

% a short form should be given in case it is too long for the running head
\titlerunning{Uzupe?niaj?ca dokumetacja projektu TAIO}

% the name(s) of the author(s) follow(s) next
%
% NB: Chinese authors should write their first names(s) in front of
% their surnames. This ensures that the names appear correctly in
% the running heads and the author index.
%
\author{Anna Zawadzka\and Sylwia Nowak\and Pavel Kuzmich\and Piotr Waszkiewicz}
%
\authorrunning{}
% (feature abused for this document to repeat the title also on left hand pages)

%
% NB: a more complex sample for affiliations and the mapping to the
% corresponding authors can be found in the file "llncs.dem"
% (search for the string "\mainmatter" where a contribution starts).
% "llncs.dem" accompanies the document class "llncs.cls".
%

\toctitle{Lecture Notes in Computer Science}
\tocauthor{Authors' Instructions}
\maketitle

\section{Opis zmian}

Podczas realizacji postawionego zadania w dużym stopniu starano się utrzymać zgodność z dokumentacją przygotowaną przed rozpoczęciem pracy. Zarówno proces generowania zbiorów słów treningowych i testowych, konwencja zapisu położenia i prędkości cząstek oraz cyfrowa reprezentacja automatu skończonego nie uległy zmianie. \\

Poprawiono jednak działanie algorytmu PSO. W przeciwieństwie do zaproponowanej wcześniej wersji nowa nie uwzględnia położenia sąsiadów w szacowaniu nowych prędkości. Obecnie wzór na wyliczenie nowej prędkości cząstki wygląda następująco: \\

v[] = v[] + c1 * rand() * (pbest[] - present[]) + c2 * rand() * (gbest[] - present[]) \\

present[] = persent[] + v[] \\

Każda cząstka podczas całego cyklu poszukiwania pamięta swoje dotychczasowe najlepsze położenie (miejsce w przestrzeni gdzie znaleziony automat miał najniższą wartość funkcji celu). Cząstka która posiada najlepsze położenie ze wszystkich cząstek zostaje cząstką global best. \\

Zmienione zostały również zasady przemieszczania się cząstek w przestrzeni. Cząstki które w danym ruchu miały wyjść poza obszar poszukiwań zamiast być stopowane zostają umieszczone na początku tego obszaru z przeciwległej strony. \\

\end{document}
